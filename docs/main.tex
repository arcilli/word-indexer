%%%%%%%%%%%%%%%%%%%%%%%%%%%%%%%%%%%%%%%%%
% Lachaise Assignment
% LaTeX Template
% Version 1.0 (26/6/2018)
%
% This template originates from:
% http://www.LaTeXTemplates.com
%
% Authors:
% Marion Lachaise & François Févotte
% Vel (vel@LaTeXTemplates.com)
%
% License:
% CC BY-NC-SA 3.0 (http://creativecommons.org/licenses/by-nc-sa/3.0/)
% 
%%%%%%%%%%%%%%%%%%%%%%%%%%%%%%%%%%%%%%%%%

%----------------------------------------------------------------------------------------
%	PACKAGES AND OTHER DOCUMENT CONFIGURATIONS
%----------------------------------------------------------------------------------------

\documentclass[12pt]{article}
\usepackage[utf8]{inputenc}
%%%%%%%%%%%%%%%%%%%%%%%%%%%%%%%%%%%%%%%%%
% Lachaise Assignment
% Structure Specification File
% Version 1.0 (26/6/2018)
%
% This template originates from:
% http://www.LaTeXTemplates.com
%
% Authors:
% Marion Lachaise & François Févotte
% Vel (vel@LaTeXTemplates.com)
%
% License:
% CC BY-NC-SA 3.0 (http://creativecommons.org/licenses/by-nc-sa/3.0/)
% 
%%%%%%%%%%%%%%%%%%%%%%%%%%%%%%%%%%%%%%%%%

%----------------------------------------------------------------------------------------
%	PACKAGES AND OTHER DOCUMENT CONFIGURATIONS
%----------------------------------------------------------------------------------------

\usepackage{amsmath,amsfonts,stmaryrd,amssymb} % Math packages

\usepackage{enumerate} % Custom item numbers for enumerations

\usepackage[ruled]{algorithm2e} % Algorithms

\usepackage[framemethod=tikz]{mdframed} % Allows defining custom boxed/framed environments

\usepackage{listings} % File listings, with syntax highlighting
\lstset{
	basicstyle=\ttfamily, % Typeset listings in monospace font
}

%----------------------------------------------------------------------------------------
%	DOCUMENT MARGINS
%----------------------------------------------------------------------------------------

\usepackage{geometry} % Required for adjusting page dimensions and margins

\geometry{
	paper=a4paper, % Paper size, change to letterpaper for US letter size
	top=2.5cm, % Top margin
	bottom=3cm, % Bottom margin
	left=2.5cm, % Left margin
	right=2.5cm, % Right margin
	headheight=14pt, % Header height
	footskip=1.5cm, % Space from the bottom margin to the baseline of the footer
	headsep=1.2cm, % Space from the top margin to the baseline of the header
	%showframe, % Uncomment to show how the type block is set on the page
}

%----------------------------------------------------------------------------------------
%	FONTS
%----------------------------------------------------------------------------------------

\usepackage[utf8]{inputenc} % Required for inputting international characters
\usepackage[T1]{fontenc} % Output font encoding for international characters


%----------------------------------------------------------------------------------------
%	COMMAND LINE ENVIRONMENT
%----------------------------------------------------------------------------------------

% Usage:
% \begin{commandline}
%	\begin{verbatim}
%		$ ls
%		
%		Applications	Desktop	...
%	\end{verbatim}
% \end{commandline}

\mdfdefinestyle{commandline}{
	leftmargin=10pt,
	rightmargin=10pt,
	innerleftmargin=15pt,
	middlelinecolor=black!50!white,
	middlelinewidth=2pt,
	frametitlerule=false,
	backgroundcolor=black!5!white,
	frametitle={Command Line},
	frametitlefont={\normalfont\sffamily\color{white}\hspace{-1em}},
	frametitlebackgroundcolor=black!50!white,
	nobreak,
}

% Define a custom environment for command-line snapshots
\newenvironment{commandline}{
	\medskip
	\begin{mdframed}[style=commandline]
}{
	\end{mdframed}
	\medskip
}

%----------------------------------------------------------------------------------------
%	FILE CONTENTS ENVIRONMENT
%----------------------------------------------------------------------------------------

% Usage:
% \begin{file}[optional filename, defaults to "File"]
%	File contents, for example, with a listings environment
% \end{file}

\mdfdefinestyle{file}{
	innertopmargin=1.6\baselineskip,
	innerbottommargin=0.8\baselineskip,
	topline=false, bottomline=false,
	leftline=false, rightline=false,
	leftmargin=2cm,
	rightmargin=2cm,
	singleextra={%
		\draw[fill=black!10!white](P)++(0,-1.2em)rectangle(P-|O);
		\node[anchor=north west]
		at(P-|O){\ttfamily\mdfilename};
		%
		\def\l{3em}
		\draw(O-|P)++(-\l,0)--++(\l,\l)--(P)--(P-|O)--(O)--cycle;
		\draw(O-|P)++(-\l,0)--++(0,\l)--++(\l,0);
	},
	nobreak,
}

% Define a custom environment for file contents
\newenvironment{file}[1][File]{ % Set the default filename to "File"
	\medskip
	\newcommand{\mdfilename}{#1}
	\begin{mdframed}[style=file]
}{
	\end{mdframed}
	\medskip
}

%----------------------------------------------------------------------------------------
%	NUMBERED QUESTIONS ENVIRONMENT
%----------------------------------------------------------------------------------------

% Usage:
% \begin{question}[optional title]
%	Question contents
% \end{question}

\mdfdefinestyle{question}{
	innertopmargin=1.2\baselineskip,
	innerbottommargin=0.8\baselineskip,
	roundcorner=5pt,
	nobreak,
	singleextra={%
		\draw(P-|O)node[xshift=1em,anchor=west,fill=white,draw,rounded corners=5pt]{%
		Question \theQuestion\questionTitle};
	},
}

\newcounter{Question} % Stores the current question number that gets iterated with each new question

% Define a custom environment for numbered questions
\newenvironment{question}[1][\unskip]{
	\bigskip
	\stepcounter{Question}
	\newcommand{\questionTitle}{~#1}
	\begin{mdframed}[style=question]
}{
	\end{mdframed}
	\medskip
}

%----------------------------------------------------------------------------------------
%	WARNING TEXT ENVIRONMENT
%----------------------------------------------------------------------------------------

% Usage:
% \begin{warn}[optional title, defaults to "Warning:"]
%	Contents
% \end{warn}

\mdfdefinestyle{warning}{
	topline=false, bottomline=false,
	leftline=false, rightline=false,
	nobreak,
	singleextra={%
		\draw(P-|O)++(-0.5em,0)node(tmp1){};
		\draw(P-|O)++(0.5em,0)node(tmp2){};
		\fill[black,rotate around={45:(P-|O)}](tmp1)rectangle(tmp2);
		\node at(P-|O){\color{white}\scriptsize\bf !};
		\draw[very thick](P-|O)++(0,-1em)--(O);%--(O-|P);
	}
}

% Define a custom environment for warning text
\newenvironment{warn}[1][Warning:]{ % Set the default warning to "Warning:"
	\medskip
	\begin{mdframed}[style=warning]
		\noindent{\textbf{#1}}
}{
	\end{mdframed}
}

%----------------------------------------------------------------------------------------
%	INFORMATION ENVIRONMENT
%----------------------------------------------------------------------------------------

% Usage:
% \begin{info}[optional title, defaults to "Info:"]
% 	contents
% 	\end{info}

\mdfdefinestyle{info}{%
	topline=false, bottomline=false,
	leftline=false, rightline=false,
	nobreak,
	singleextra={%
		\fill[black](P-|O)circle[radius=0.4em];
		\node at(P-|O){\color{white}\scriptsize\bf i};
		\draw[very thick](P-|O)++(0,-0.8em)--(O);%--(O-|P);
	}
}

% Define a custom environment for information
\newenvironment{info}[1][Info:]{ % Set the default title to "Info:"
	\medskip
	\begin{mdframed}[style=info]
		\noindent{\textbf{#1}}
}{
	\end{mdframed}
}

% @arcilli
\renewcommand\refname{}
\lstset{frame=tb,
	language=Java,
	aboveskip=3mm,
	belowskip=3mm,
	showstringspaces=false,
	columns=flexible,
	basicstyle={\small\ttfamily},
	numbers=none,
	numberstyle=\tiny\color{gray},
	keywordstyle=\color{blue},
	commentstyle=\color{dkgreen},
	stringstyle=\color{mauve},
	breaklines=true,
	breakatwhitespace=true,
	tabsize=3
}
\usepackage{listings}
\usepackage{color}
\definecolor{dkgreen}{rgb}{0,0.6,0}
\definecolor{gray}{rgb}{0.5,0.5,0.5}
\definecolor{mauve}{rgb}{0.58,0,0.82}
\renewcommand{\contentsname}{Cuprins}
\renewcommand*{\algorithmcfname}{Algoritm}
\usepackage{graphicx}
\graphicspath{ {./images/} }
\usepackage{indentfirst} % Include the file specifying the document structure and custom commands
%----------------------------------------------------------------------------------------
%	ASSIGNMENT INFORMATION
%----------------------------------------------------------------------------------------

\title{Regăsirea informațiilor pe web: indexare \& căutare} % Title of the assignment

\author{Gabriel Răileanu\\ \texttt{1409A}} % Author name and email address

\date{aprilie 2020} % University, school and/or department name(s) and a date

%----------------------------------------------------------------------------------------

\begin{document}

\maketitle % Print the title

%----------------------------------------------------------------------------------------
%	INTRODUCTION
%----------------------------------------------------------------------------------------
%\newpage
%\tableofcontents
%\newpage

\section{Descrierea problemei}
Proiectul de față își propune indexarea unui set de documente și implementarea unei interfețe care să permită utilizatorului căutarea de termeni în colecția de intrare.
Pe parcursul proiectului au fost urmărite următoarele scopuri:
\begin{itemize}
	\item obținerea unei procesări adecvate a cuvintelor determinate în cadrul unui text
	\item studiul bazelor de date non-relaționale și a diferitelor mecanisme de stocare
	\item implementarea de metode  folosite în operațiile de căutare
\end{itemize}
Pentru a executa o căutare a noțiunilor ce prezintă interes, setul de documente a necesitat procesări anterioare, la finalul cărora a avut loc operația de căutare. Etape:
\begin{enumerate}
	\item filtrarea termenilor de interes din colecția inițială
	\item aducerea cuvintelor la forma normală
	\item calcul index direct cantitativ \& index invers cantitativ
	\item căutarea efectivă
\end{enumerate}
\section{Soluția propusă}
\subsection{Filtrarea termenilor}
Pentru filtrarea cuvintelor s-au folosit liste de excepții și de \textit{stopword}-uri specifice limbii engleze.
\subsection{Forma normală}
Pentru a ajunge la forma normală a cuvintelor, s-au avut în vedere doi algoritmi specifici: Porter și Lovins. După o analiză comparativă a celor doi algoritmi, s-a folosit o implementare bazată pe algoritmul lui Lovins. Acesta este mai rapid, făcând un compromis pentru spațiul necesar. În linii mari, algoritmul folosește seturi de sufixe, aplicând pe cuvintele de bază reguli (conține 294 de sufixe, 29 de condiții și 35 reguli de transformare).
\subsection{Indecși}
Pentru a ajunge la scopul final al proiectului (realizarea de operații de căutare pe colecția inițială), s-a realizat 2 tipuri de indecși:
\begin{itemize}
	\item index direct cantitativ
	\item index invers cantitativ
\end{itemize}
Fiecare astfel de index este persistat într-o bază de date non-relațională (în acest caz alegându-se o bază de date orientată document: MongoDB).
\subsubsection{Index direct cantitativ}
Fiecare document din colecția de indexat este parcurs caracter cu caracter. Caracterele luate în considerare pentru contorizat sunt alcătuite fie din litere, fie din cifre.
Astfel, la sfârșitul procesului de indexare directă, pentru fiecare document, în baza de date vom avea persistate un număr de intrări egal cu numărul de documente din colecția de indexat, documentele de forma:
\begin{center}
$(k,\{w:n\}) \longrightarrow <(w_{1}, v_{1}), (w_{2}, v_{2}), \dots, (w_{n}, v_{n})>$
\begin{tabular}{r@{: }l r@{: }l}
	$k$ & numele documentului & $w_{1..n}$ & cuvânt\\
	$v$& nr. apariții & $n$ & nr. cuvinte din document
\end{tabular}
\end{center}
Pentru varianta \textbf{paralelă} a implementării, s-a folosit un \textit{pool} (bazin) de workeri, thread-uri ale procesului master. Procesul master trimite către workeri câte un fișier, pe care acesta îl indexează. La finalul procesului, dacă mai există documente care nu au fost indexate, worker-ul mai primește un fișier pentru indexare. 
\subsubsection{Index invers cantitativ}
În implementarea indexului invers cantitativ s-au folosit rezultatele indexului direct cantitativ, concatenându-se rezultatele. Indexul invers cantitativ are forma:
\begin{center}
	$(w,\{doc:n\}) \longrightarrow <(w, [\{doc_{1}, n_{1}\}, \{doc_{2}, n_{2}\}, \dots, \{doc_{m}, n_{m}\}])>$
	\begin{tabular}{r@{: }l r@{: }l}
		$w_{1..n}$ & cuvânt & $doc_{i}$ & numele documentului\\
		$n_{i}$& nr. apariții & $m$ & nr. documente
	\end{tabular}
\end{center}
Pentru varianta \textbf{paralelă} a implementării, fiecare worker va primi un document (a cărui index direct este deja creat în etapa precedentă) și va calcula indexul invers, pe care îl persistă în baza de date cu o funcționalitate de \textit{update or set} (upsert). Astfel, se evită \textit{data race}-ul și concurența asupra aceleeași intrări în baza de date.
\par
În cazul în care în indexul invers apar mai multe intrări pentru aceeași cheie (aici, cuvântul de căutat), căutarea se efectuează tot cu o complexitate redusă, datorită indexului pe text pe care îl pune la dispoziție MongoDB. Astfel, problema se reduce la regăsirea tuturor intrărilor cu acea cheie, lucru care nu presupune decât adăugarea de logică în aplicație.
\subsection{Căutarea}
\subsubsection{Căutarea booleană}
Căutarea booleană returnează adevărat sau fals, în funcție de găsirea sau nu a termenului într-o colecție de documente.
\par
Se pot forma, astfel și propoziții pentru căutarea cuvintelor care apar împreună sau a unor grupuri de termeni în care să apară cel puțin unul sau în care să nu apară niciunul dintre termenii căutați.
\subsubsection{Căutarea vectorială}
Căutarea vectorială își propune inducerea unei ordini în rezultatele oferite de căutarea booleană. În implementarea de față s-a utilizat o distanță de tip cosinus.
\par
Astfel, a fost calculat vectorul asociat \textit{query}-ului introdus de utilizator, apoi vectorii asociați fiecărui document. Distanța cosinus a fost calculată pentru fiecare pereche $<query, document>$, iar apoi documentele ce constituie rezultatul căutării au fost returnate în ordinea dată de distanța cosinus.
\section{Concluzii}
Proiectul a prezentat principalele componente ale unui motor de căutare și etapele prin care un text este procesat pentru a realiza o căutare a cunoștințelor optimă.
\newpage
\section{Bibliografie}
\begin{thebibliography}{9}
	\bibitem{hadoopInAction} 
	Regăsirea informațiilor pe web: notițe de curs.
	
	\bibitem{HDFSBlocks} 
	Regăsirea informațiilor pe web: lucrări de laborator.
\end{thebibliography}
\end{document}
